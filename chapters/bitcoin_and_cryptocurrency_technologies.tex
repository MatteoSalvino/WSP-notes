\section{Introduction to Bitcoin}
When we need to buy something on the web typically we use the SSL protocol and a credit card. This approach is very simple, because doesn't need any specialized software, it's compliant with the credit card mechanism and it's one of the most used method for paying in the web. However, using this approach we reveal our information to potential malicious sellers and these increases the number of possible disputes. Furthermore, this approach is an expensive method for the shop that sells the the item of interest of the user and for sure the transaction for buy such item is not anonymous but is correctly tracked by the credit card company. For this reason, many credit card companies proposes a mechanism called \textbf{Secure Electronic Transactions (SET)}. We want to underline that this system never became operative for several reasons : all users must have a certificate for guaranteeing their payments and was too complex to put in place from the requirements standpoint both for shops and users. Few years later Satoshi Nakamoto developed \textbf{Bitcoin (BTC)}, which is a digital coin that beyond the money characteristics has the two following additional properties : it's decentralized (absence of a central authority) and it's immune to sovereign censorships. We can see Bitcoin as a combination of a currency, a payment system and a collection of algorithms and software implementation. The primary goal of the Bitcoin inventor was to make a system with low transaction costs and anonymous. In march $2017$ one bitcoin is worth about $\$ 1141$. There are slightly more $15$ million bitcoins, but the algorithm has been implemented in such a way that won't be no more than $21$ million bitcoins if the protocol doesn't changes. Now the question is how can we use Bitcoin ? First of all we need to visit bitcoin.org website, and we create a wallet which has associated an ID. Then using a proper software for managing our wallet, it will create public/private key pairs for us as needed. For each pair, there is a corresponding bitcoin address, which is a $160$-bit hash of the public key. Then we send the bitcoins to the recipient address and we sign the transaction with our private key, so that the recipient can check our identity. Notice that a transaction might also include a small transaction fee. Bitcoin uses the following cryptographic tools :
\begin{itemize}
\item \textbf{Hash functions} :
\item \textbf{Digital signatures} :
\item \textbf{Public key} : 
\end{itemize}
Now, we want to understand why designing a electronic cash is so hard. The first issue that we could face goes under the name of \textbf{double spending}. Suppose Alice that has $X$ bitcoins and sends a message to Bob says "ok, transfer $X$ bitcoins to Bob". Next, Alice at the same time sends also a message to Charlie to transfer to him the same amount of bitcoins. In other words Alice is trying to double spending its $X$ bitcoins. Now, if there is a bank for handling this problem is very easy, but at this point we are at the starting point again because the bank knows everything and it's working for us and should be paid. A possible solution to avoid this problem is to maintain a block chain as a public ledger of all transactions. So when Alice wants to adds a block to this ledger for sending $X$ bitcoins to Bob, the operation is permitted. Whereas, when Alice in a future time point, will try to add a block correspondent to the transactions towards Charlie, the operation will not be allowed, because there is a check that notice that Alice is not the owner of such amount of bitcoins anymore. Typically in bitcoin, each block contains several transactions and a new one is created every $\sim 10$ minutes. Naturally each block should be coherent, i.e. coherence means that two transactions for the same amount of bitcoins from the same user should belongs to the same block, otherwise the Bitcoin community will easily notice that that user is trying to double its bitcoins. So the idea is to make this public ledger as the central authority for managing and checking all the transactions. 