\section{Privacy in the electronic society}
Privacy is the ability of an individual or group to seclude themselves or information about themselves and thereby express themselves selectively (Wikipedia definition). However, privacy is a very complex concept to define, because there exist several interpretations of such concept. An issue that has been considered quite often in the past is called \textbf{nothing to hide}. The basic idea is that, in the past privacy issue was essentially a difficulty coming from the government, in the sense that the government was interested in information about people. The idea is that if we have nothing to hide, we don't care about the fact that the government is interested in information about us. However, this position is not probably completely correct wrt what happened in many societies and especially what happens now in the digital society. The problem is that if we have nothing to hide, what do we have to fear ? The problem is that here there is a major concern about the role of governments, where they are interested in information about people also for purposes that are not known. The main issue is that privacy is not to protect the presumptively innocent from true but damaging information, but rather to protect the actually innocent from damaging conclusions drawn from misunderstood information. The nothing to hide concept has one weakness because focus on the information associated with state security. It doesn't consider issues such as :
\begin{itemize}
\item \textbf{Aggregating data} : it's a combination of small bits of innocuous data, that when aggregated reveals useful things.
\item \textbf{Exclusion} : it's a situation when the people don't have the knowledge about how their information are used and cannot alter wrong data.
\item \textbf{Secondary use} : it represent the scenario that data are collected for one specific purpose, but they could be used for other purposes.
\end{itemize}
The issue of privacy nowadays is more complicated, since we have multiple concepts to clear such as personhood, intimacy, secrecy, limited access to the self and control over the information. Naturally, this issue arise in several context and aspects of human life such as data collection by companies, different social contexts (family, social networks) and control of private information against the collection of private information. In our desire of privacy we can distinguish two aspects as human beings : the first one is that our desire of privacy is evolving depending on the situation, but on the other side we are really willing to give our privacy for an immediate reward or because we do not see the dangerous of privacy for us but we will see for others. In particular, privacy can be seen as it is constituted by four components :
\begin{itemize}
\item \textbf{Solitude} : it's a scenario where an individual is separated from the group and freed from the observation of other persons.
\item \textbf{Intimacy} : in this case an individual is part of a small group.
\item \textbf{Anonymity} : it's a situation where an individual is public but still seeks and find freedom from identification and surveillance.
\item \textbf{Reserve} : this refers to the creation of a psychological barrier against unwanted intrusion holding back communication.
\end{itemize}
The idea is to limit access to the information, since there are laws that prohibit the collection and disclosure of such data, and technological tools that facilitates anonymous transactions in order to minimize the disclosure of information. Another important point is that we want have control about our information, but this is facilitated thanks to the technology that gives us informed consent, keeping track of and enforcing privacy preferences. In the market we have \textbf{privacy policies} that allows the consumers know about site's privacy practices. So consumers can decide to accept or reject these policies, when to get in or get out and who to do business with. The major advantage of the presence of these policies is that they increases the consumer trust. However, they have also some problem such as they are often difficult to understand, hard to find, take a long time to read and changes without explicitly noticing their consumers. In particular, the cost of reading privacy policies is typically related to how difficult is reading them. In fact, typically they are constituted by several pages of very small characters. For example, if everyone read the privacy policy for each site they visited once per year the estimated cost in hours is about $244$ hours per year, which is equivalent to approximately $\$3534$ per year. What the companies typically does with our information is not make them public, so our privacy is preserved, but they will use them to send specific advertisements to us.